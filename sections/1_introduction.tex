\section{Introduction}
\label{sec:introduction}

% 5 keys of Introduction-------------------

%1.What was I investigating? (problem)
%2.Why was it important? 
%3.What was already know about the subject of my research?
%4.What did I expect to know after doing the research?
%5.How did I  approach the problem?
%-------------------------------------------

%Why ESS is needed.
\PARstart{F}{or} priority areas of clean energy use possesses sufficient quantities of alternative energy to avoid air pollution emission, renewable energy sources (RESs) have increasingly received attention and is by now widely accepted to conventional power systems \cite{b1,b2,b3,b4,b5-wind-data,b6}. Meanwhile, Krabi, which is a southern province of Thailand, is projected to possibly reach around 100\% of electricity consumption using RES in 2026 \cite{b7}. Nevertheless, the characteristics of volatile renewable sources probably lead to an excess of energy production that would be wasted if the demand of consumers and the supply of generated energy does not balance. For instance, a global quantity of curtailed electrical energy of 940.8 billion kWh was calculated in the year 2013 \cite{b9}. Combining renewable and traditional energy resources as well as energy storage systems (ESSs) known as hybrid renewable energy systems (HRESs) is recently appeared to respond with the increasing use of RESs as alternatives to conventional power generation systems.

\par

%1.What was I investigating? (Problem)
% ESS Sizing determination considering the barrier of n long-term investment cost and short-term operational conditions is an important whether any method you use (according to the other researches)
In front of the meter, however, high penetration of volatile renewable sources, which generally depends on weather conditions, probably affects on dispatch-able RES \cite{b10}; thus reducing the deviation is unavoidably needed. To increasing in the power distribution networks, ESSs are suggested to cope with downsides of RESs, which largely depend on the flexibility of resource \cite{b11,b12}. Generally, ESSs have been adopted for mitigating the RES integration problem by reducing the deviation, and for solving other problems such as stabilizing voltage fluctuation (seconds of run-time), smoothing the outputs (minutes of run-time), and providing spinning reserves or ancillary services (hours of run-time) \cite{b14}. However, the barrier for ESSs deployment is optimizing between long-term investment cost and short-term operational conditions \cite{13-main_barrier_ESS}. 

\par
% 2.Why was it important? 
%*** Measuring how far from error distribution is better than measuring the forecasting accuracy in terms of optimal sizing of ESS.



% บอกว่าการตรวจดูforecasting accuracyเพื่อ หา optimal size of ESS อย่างเดียวไม่เพียงพอ ควรที่จะต้องดู character / component ของ time series data set ด้วย


%4.What did I expect to know after doing the research?
%คาดว่าจะนำเสนอ indicator ที่ใช้ในการตวรแต่ละ component ของ time series แยกตามประเภทของ RES โดยเทียบกับ Probability Error Distribution (Mixed distribution function; obtained based on weighted normal and laplace distribution) ที่ได้ตีพิมพ์ไปแล้ว 
%ถ้า component ไหนให้ค่า error distribution ได้ใกล้เคียงกัย Mixed distribution ก็หมายถึงว่ามันให้ benefit ต่อการ optimize size of ESS มากเท่านั้น

%4.What did I expect to know after doing the research?

%*** Time series components (seasonal trend ...) which one important (ranking them)

%*** Measuring each component how far from error distribution is better than measuring the forecasting accuracy in terms of optimal sizing of ESS may contribute the new indication (Just think about how to measure it)


%5.How did I  approach the problem?

\subsection{Subsection}


